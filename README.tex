% Created 2025-05-27 Tue 10:30
% Intended LaTeX compiler: pdflatex
\documentclass[11pt]{article}
\usepackage[utf8]{inputenc}
\usepackage[T1]{fontenc}
\usepackage{graphicx}
\usepackage{longtable}
\usepackage{wrapfig}
\usepackage{rotating}
\usepackage[normalem]{ulem}
\usepackage{amsmath}
\usepackage{amssymb}
\usepackage{capt-of}
\usepackage{hyperref}
\usepackage{mathpartir}
\usepackage{amsthm}
\usepackage{hyperref}
\author{Dowland Aiello}
\date{5/26/25}
\title{Strong Normalization of the Simply-Typed Lambda Calculus in Lean by Decomposition Into the SK Combinators}
\hypersetup{
 pdfauthor={Dowland Aiello},
 pdftitle={Strong Normalization of the Simply-Typed Lambda Calculus in Lean by Decomposition Into the SK Combinators},
 pdfkeywords={},
 pdfsubject={},
 pdfcreator={Emacs 30.1 (Org mode 9.7.27)}, 
 pdflang={English}}
\begin{document}

\maketitle
\tableofcontents

\section{Abstract}
\label{sec:orgd36852f}

Proofs of strong normalization of the simply-typed lambda calculus have been exhaustively enumerated in the literature. A common strategy invented by W. W. Tait known as "Tait's method," (Robert Harper, 2022) interprets types as sets of "well-behaving" terms which are known to be strongly normalizing and composed of expressions in some such set.
Strong normalization of the typed SK combinator calculus has been comparatively under-studied. Herein, I demonstrate that the typical proof of strong normalization using Tait's method holds for the typed SK combinator calculus. I also show that decomposition of the STLC into the SK combinator calculus simplifies the typical proof of strong normalization.
\section{A Type Discipline for the SK Combinators}
\label{sec:org5de6060}

I consider the usual SK combinator calculus defined as such:

\begin{align}
& K xy = x \\
& S xyz = xz (yz)
\end{align}

A natural interpretation of the combinators as typed functions results in the dependent typing:

\[
\inferrule
  { \Gamma \vdash A : K \ \Gamma,x : A \vdash B : L }
  { \Gamma \vdash (\forall x : A.B) : L}
\]
\[
\inferrule
  { }
  { \Gamma T_{n} : T_{n + 1} }
\]
\[
\inferrule
  { \Gamma \alpha : T_{n}, \beta : T_{m}, x : \alpha, y : \beta }
  { \Gamma \vdash K : (\forall x, y.\alpha) }
\]
\[
\inferrule
  { \Gamma \alpha : T_{n}, \beta : T_{m}, \gamma : T_{o}, x : (\forall x : \alpha, y : \beta.\gamma), y : (\forall x : \alpha.\alpha), z : \alpha }
  { \Gamma \vdash S : (\forall x, y, z.\gamma) }
\]
\section{Decomposition of the Simply-Typed Lambda Calculus into Dependently Typed SK Combinators}
\label{sec:org6907ff3}

I utilize an SK compilation scheme outlined in "The Implementation of Functional Programming Languages" (Peyton Jones, Simon L., 1987):

\begin{align}
(\lambda x.e_{1}\ e_{2})\ arg &= S (\lambda x.e_{1}) (\lambda x.e_{2})\ arg \\
(\lambda x.x) &= SKK \\
(\lambda x.c) &= K c
\end{align}

I consider a generic simply-typed lambda calculus with base types \(B\), a type constructor \(\rightarrow\) and the type universe:

\[
T = \{ t \mid t \in B\}\ \cup\ \{ t \mid \exists\  t_{1} \in T, t_{2} \in T, t = t_{1} \rightarrow t_{2} \}
\]
\subsection{Type Expressivity \& Equivalence}
\label{sec:org7d48d09}

I define a mapping (M\textsubscript{t}) from the \(\rightarrow\) type constructor to \(\forall\): \((\alpha \rightarrow \beta) \mapsto \forall x : \alpha.\beta\). I also assume the existence of a mapping (M\textsubscript{c}) from the base types \(B\) to arbitrary objects in my dependently-typed SK combinator calculus. Type inference is trivially derived from the above inference rules: \(\forall c \in B, \exists\ t, t', c : t \implies t' = M_{t} t \implies M_{c} : t\).

It follows that every well-typed expression in our simply-typed lambda calculus has an equivalent SK expression:

\begin{proof}
Assume (1) that for all $c \in B, \exists!\ c' \in M_{c}, c' = M_{c} c$. Assume (2) that for all $\{t_{1}, t_{2}, t\} \subset T, t = (t_{1} \rightarrow t_{2}), \exists!\ t' \in M_{t}, t' = M_{t} t$. Per above[[decomplemma:1]], and induction on (1) there exists a mapping from every lambda expression to an SK combinator expression. It follows by induction on $e : t$, where e is well-typed per the inference rules[[decomplemma:1]] that all $t \in$ the simply-typed $T$ are in $M_{t}$.

\end{proof}
\section{Proof}
\label{sec:org680014b}
\subsection{Strong Normalization of the Typed SK Combinators}
\label{sec:orgc5723d8}
\subsection{Strong Normalization of the STLC}
\label{sec:org8c074fe}
\subsection{Encoding in Lean}
\label{sec:org5397e5e}

\noindent
Peyton Jones, Simon L. (1987). \emph{The Implementation of Functional Programming Languages (Prentice-Hall International Series in Computer Science)}, Prentice-Hall, Inc..

\noindent
Robert Harper (2022). \emph{How to (Re)Invent Tait’s Method}.
\end{document}
